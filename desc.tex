\documentclass{article}
\setlength{\topmargin}{-.7in}
\setlength{\textwidth}{6.9in}
\setlength{\textheight}{9.5in}
\setlength{\oddsidemargin}{-.16in}
\setlength{\evensidemargin}{-.16in}
\newcommand{\comment}[1]{}
\begin{document}

\newcommand{\LB}{$[$}
\newcommand{\RB}{$]\;$}
\newcommand{\LP}{$($}
\newcommand{\RP}{$)\;$}

\begin{center}
{\Large \bf CSE 504  Project}\\[0.1in]
\end{center}
%\section{Description}
In this project, you will complete the rest of the E-{}- compiler. Your
target language will be an abstract assembly language that is
further described in Section~\ref{targetlang}. It is a 32-bit
machine that has the following characteristics:
\begin{itemize}
\item 4 giga words of memory, where each word can store any value of
  primitive type such as float or int
\item a large number of integer and floating point registers,
\item supports move operations from memory to registers and vice-versa,
\item supports conditional and unconditional branches (both direct and
  indirect branches), and arithmetic/logical operations (but not
  call/return),
\item supports an operation to fetch the next event
\item requires all operands to be in registers (except for move operations).
\end{itemize}
%
I am providing you the assembler to convert your abstract assembly code
into machine code. It works by translating assembly code into C, and then
uses gcc to translate this C-code into an executable. (Although this may
seem convoluted, it was easy to implement and works well.)

This assignment should be completed in groups of 3 to 4 students.
You will start with your code for the typechecker assignment, and
add methods to perform the following tasks:
\begin{description}
\item[Type-checking:] Implement a type-checker for E--. A description of
  the type checker and sample tests are provided to you. While the tests are
  designed to be helpful to you, you need not try to match the exact outputs
  shown in the sample output file; but use the sample outputs to ensure
  that your type checker works correctly.
\item[Memory allocation:] Allocate storage for global and local variables,
   as well as temporaries needed. 
\item[Implementing functions:] Implement procedure calls and returns,
as well as parameter passing.
\item[Compilation of event-matching:] You can see that E-- uses patterns
on events that look very much like regular expressions. You need to 
compile these patterns into code for performing these pattern matches.
\item[(Intermediate) code generation:] Generate abstract machine code. 
\item[Optimization:] I expect you to implement some optimizations, although
the exact ones are up to you. 
\end{description}
%
If you implement pattern matching or optimization, then those components
will be evaluated on the basis of how complete those features are, as
well as the speed of execution.

We will make some minor modifications to E-- before proceeding on this
assignment. The most important one will be the inclusion of a while loop,
and a break statement. While loops could be nested, and can contain
break statements of the form ``break $n$'' that cause control to be
transferred to the statement immediately following the end of the
$n$th enclosing while-loop. 

\section{Timeline}
%
This project will be done by groups of 3 to 4 students. Here are the
deadlines:
\begin{description}
\item[10/24:] Project posted.
\item[10/28:] Project groups to be finalized, and the names sent to the TA.
\item[11/14:] Interim report due. By this date, you should have completed
  the detailed design of your entire project. You should also have finished
  implementing type checking and memory allocation,
\item[12/10:] Final report and project demos due. The demos may be scheduled in
  the following week.
\end{description}

\section{Grading}
%
Unlike the programming assignments you have been working on so far, where
the tasks were specified in great detail and with specificity, a project is
much more ``free form.'' I expect to see significant variation between
groups in terms of what they deliver, and how well (and fast) it works. 

Groups are expected to show a significant deal of imagination and creativity
in their projects. I will award up to 10 bonus points for groups that
surpass my expectations in this regard.

Finally, a description of type-checker assignment from previous offerings
of the course is being provided to you in the hope that it will be
helpful to you. As mentioned earlier, you need not follow it exactly,
but do make sure that your type checker is implemented correctly.

We will follow the same submission requirements that we have been following so
far for the assignments. The group leader will submit the project implementation
on the Blackboard. He/she will identify the names and the IDs of the project
members during submission. Project members do not need to submit anything to the
Blackboard.

In addition to submitting your project on Blackboard, you
will need to give a demo. Your final grade will depend
on both. 

\section{Target Machine} \label{targetlang}
%
The intermediate code is intended to model a very simple instruction set.
Complete information about the instruction set can be obtained from the source
code files for the assembler. In particular, note that each icode instruction is
translated by the assembler into a macro with the same name. The definitions of
these macros appear in \verb+E--_RT.c+, and should provide you a clear
understanding of the instruction set semantics.

Icode supports 1000 integer registers (R000 to R999) and 1000 floating point
registers (F000 to F999). 

Icode memory starts at address zero, and goes on to a maximum address specified
as a command-line option to the C-program generated by the assembler. (The
default memory size is 1M words.) Memory is word-addressed, not byte-addressed.
This memory is represented as an array in the C-program. Guard zones are set up
on either side of the array to catch out-of-bounds accesses (which will trigger
a memory exception). To simplify the organization of memory, we make the
assumption that \verb+sizeof(int) == sizeof(float)+. This enables integers and
floating point numbers to be both stored in one memory word.

Execution of icode starts at the first instruction in the icode file, and
execution is stopped when control flows past the last instruction in the file.

The instructions can be divided into the following categories. In all cases,
destination operands are listed following the source operands. Any instruction
can have a label.
%
\begin{itemize}
\item {\em integer arithmetic and bit operations:} includes ADD, SUB, DIV,
  MUL, MOD, NEG, AND, OR and XOR. The last 3 operations are bit-wise
  operations, applying the specified boolean operation on corresponding
  bits of two integer operands. These operations have two source operands
  (just one source in the case of NEG) that can either be a value or a
  register, and a destination operand that must be a register.
%
\item {\em floating point arithmetic:}
      includes FADD, FSUB, FDIV, FMUL and FNEG. They have two source operands
      (one in the case of FNEG) that can be values or registers, and a 
      destination operand that must be a register.
%
\item {\em integer relational operations:} includes GT and GE that treat their
      two operands as signed integers; UGT and UGE that operate on unsigned 
      integers; EQ and NE that operate on integers regardless of size; 
      All operands can be values or registers.

      NOTE: Relational operators are not stand-alone instructions, but 
      instead, appear as part of a conditional jump instruction.
%
\item {\em floating point relational operations:} includes FGT, FGE, FEQ and FNE 
      that operate on floating point operands (values or registers).
%
\item {\em print instructions:} include PRTI, PRTS, and PRTF that each take a 
      a single operand. In the case of PRTI, this operand represents the 
      integer value to be printed, or the register that needs to be printed. 
      PRTF is similar, except that it prints a floating point operand.
      In the case of PRTS, the operand is a string constant (enclosed within
      double quotes) or a register containing a string constant, i.e., the
      register was previously initialized by a \verb+MOVS <string_constant> <reg>+. 
%
\item {\em jump instructions:}
\begin{itemize} 
  \item {\em unconditional jump:} \verb+JMP <loc>+, where \verb+<loc>+ is a label. 
  \item {\em conditional jump:} \verb+JMPC <cond> <loc>+, where \verb+<cond>+ is a relational
         operator with parameters. Example: \verb+JMPC GE R000 R001 <loc>+
  \item {\em indirect jump:} \verb+JMPI <reg>+, where \verb+<reg>+ is an integer register that has previously been initialized with a label value using a MOVL
         instruction.
  \item {\em conditional indirect jump:} Example \verb+JMPCI FGE F001 1.0 R010+.
\end{itemize}
%
\item {\em data movement instructions: }
  \begin{itemize}
    \item {\em move label to a register:} \verb+MOVL <label> <intreg>+
    \item {\em move string to register:}  \verb+MOVS <stringConstant> <intreg>+

    \item {\em move integer to register:} \verb+MOVI <valueOrReg> <intreg>+
    \item {\em move float to register:}   \verb+MOVF <valueOrReg> <freg>+

    \item {\em move int to float reg:}    \verb+MOVIF <intreg> <freg>+
    \item {\em move float to int reg:}    \verb+MOVIF <freg> <intreg>+

    \item {\em load int reg from mem:}    \verb+LDI <reg> <valuOrReg>+
    \item {\em load float reg from mem:}  \verb+LDF <reg> <valuOrReg>+
    \item {\em store int reg to mem:}     \verb+STI <reg> <valuOrReg>+
    \item {\em store float reg to mem:}   \verb+STF <reg> <valuOrReg>+
  \end{itemize}
%
\item {\em input instruction:} (for reading input data)
      \verb+IN <reg>+ reads a single byte from input stream and stores it into
         the specified register. A negative return value indicates an
         error, with the semantics the same as that of getc.
      \verb+INI <reg>+, \verb+INF <reg>+ read an integer (or floating point number) into the 
         specified register. Aborts execution if any errors are encountered.
\end{itemize}

\section{Event Matching}
%
In a full implementation of the E-{}- language, event matching should be
implemented using one of the direct DFA construction techniques: either
the one based on the concept of derivatives discussed in class, or the
technique described in your textbook. Of course, neither of these
algorithms handle event parameters. You can refer to Section 5 of the
paper available at {\tt
  http://seclab.cs.sunysb.edu/seclab/pubs/usenix99.pdf} to understand how
event arguments can be handled. (Note that there are some notational
differences: the paper uses ``;'', ``$||$'' and ``$*$'' in the places of
``\verb+:+'', ``\verb+\/+'' and ``\verb+**+''. The paper uses the term
``REE'' to refer to these event patterns, and NEFA to the automata
constructed for matching these patterns.)

As mentioned before, you are not required to implement event
patterns in this assignment. However, you may decide that you
want to implement it any way. (If you do this, then you can
skip some other component of E-{}- implementation, specifically,
the implementation of functions and parameter passing. Moreover,
you will not need to implement event arguments.)

Events will be input using the IN instruction in your assembly. You can
assume that, for the purposes of this project, all event names will have
only a single character, so it will be easy to input event names using the
IN instruction that returns just a single byte. (This limits us to a
maximum of 53 events.) For this assignment, we will restrict
ourselves to integer and floating point event arguments. Since you have
the declaration of events, which specifies the number and types of event
arguments, you should be able to use an appropriate number of IN
operations to input event arguments and convert them into integers or
floating point numbers. For instance, given the following declaration

\begin{verbatim}
event a(int x1, float x2)
\end{verbatim}

the input
\begin{verbatim}
aK\0\0\0\0\00AaP\0\0\0\0\0@A
\end{verbatim}
will denote a sequence of two events \verb+a(75,11.0),a(80,12.0)+. (Note
that \verb+\0+ denotes a null character.) Note that the input
representation reflects how integers and floating point numbers are
represented internally. For instance, an integer is represented using
4-bytes, with the first byte representing the least significant byte of
the integer. A similar observation applies to floating point numbers,
except that their internal representation is a bit more complex. The above
event information can be read using the following icode:

\begin{verbatim}
IN R100 // Read event name
....    // Decide how many parameters to read, and their types
INI R101 // Read the integer parameter
INF F101 // Read the floating point parameter
\end{verbatim}

Note that \verb+IN+ behaves differently from \verb+INI+ and \verb+INF+ 
on errors: \verb+IN+ will return a negative value to denote input
errors or end-of-file. The other two instructions will simply abort the
program with an error message.
%
\section{Project Path}
%
Most of your implementation effort in this assignment will be concentrated
in two functions that you add to subclasses of AstNode and SymTabEntry,
namely, \verb+memAlloc+ and \verb+codeGen+. The former function is
responsible for allocating memory: it will traverse the program to identify
global variables and associate them with memory locations where they will
be stored. The allocator will also need to determine the offsets on
the activation records for parameters and local variables. Finally,
it will need to compute the number of temporary variables needed,
and allocate them. (Temporary variables need to be registers or local
variables. If you use registers, then make sure that you do this only
if you can bound the number of registers needed --- for instance, if you
have an expression that contained a few thousand operators, even 1000
registers would not be enough.)

Assembly code will be generated by the \verb+codeGen+ function.

You will need to construct appropriate test cases to test your
code. Some of the key features that you should test are:
\begin{itemize}
\item type checking: make sure that typing errors and coercions are handled
  correctly, and that implicit and explicit typings are handled.
\item event pattern matching, if implemented. Make sure that you
  test patterns that contain nontrivial combinations of closure,
  sequencing, and alternation operations.
\item function calls and returns. You should ensure that the code
  works with different types and numbers of parameters. You should
  also make sure that recursion works correctly.
\item if-then-else, loops and break statements.
\item large expressions that test your technique for 
  allocating and managing temporaries
\end{itemize}

In addition to submitting your project on Blackboard, you
will need to give a demo. Your final grade will depend
on both. 
%
\section{Example program in assembly code}
%
\small
\begin{verbatim}
JMP begin
// Test function
prt3: ADD R900 1 R901
LDI R901 R050 // Return address now in R050
ADD R901 1 R901
LDF R901 F051 // param 1
ADD R901 1 R901
LDI R901 R052 // param 2
ADD R901 1 R901
LDF R901 F053 // param 3
ADD R901 1 R901
LDI R901 R054 // param 4
PRTS R054
PRTF F053
PRTS "*"
PRTI R052
PRTS "="
PRTF F051
PRTS "\n"
ADD R900 5 R900
JMPI R050
// 
// Here is where the main program starts
//
// Basic integer operations
begin: MOVI 1036 R000
MOVI 1172 R001
MOVI 1169 R002
SUB  R001 R000 R003
SUB  R001 R002 R004
ADD  R004 R003 R005
MUL  R003 R004 R006
DIV  R006 R004 R007
MOD  R001 R000 R008
NEG  R008 R008
// Basic FP operations
MOVF 1.036 F000
MOVF 1.172 F001
MOVF 1.169 F002
FSUB F001 F000 F003
FSUB F001 F002 F004
FADD F004 F003 F005
FMUL F003 F004 F006
FDIV F006 F004 F007
FNEG F007 F007
// Basic logical operations
MOVI 1036 R000
MOVI 1048 R001
MOVI 1052 R002
OR  R001 R000 R003
AND R001 R002 R004
XOR R001 R002 R004
// A simple loop
IN R010
ADD R010 1 R010
SUB R010 1 R020
IN R021
MOVIF R021 F022
FDIV F022 100.0 F022
IN R021
MOVIF R021 F021
FADD F021 F022 F021
MOVF 0.0 F011
L1: SUB  R010 1 R010
JMPC GE 0 R010 L2
FADD F021 F011 F011
JMP L1
// A simple function call
L2: PRTF F011
PRTS "\n"
MOVI 10000 R900 // init SP
MOVS "Computed using loop that " R022
STI R022 R900
SUB R900 1 R900 // pushed param 4
STF F021 R900
SUB R900 1 R900 // pushed param 3
STI R020 R900
SUB R900 1 R900 // pushed param 2
STF F011 R900
SUB R900 1 R900 // pushed param 1
MOVL L3 R025
STI R025 R900
SUB R900 1 R900 // pushed return addr
JMP prt3
L3: PRTI R900
PRTS "\n"
// Input two events: first arg of event is int, second is float
IN  R201
PRTI R201
PRTS "("
INI R200
PRTI R200
PRTS ", "
INF F200
PRTF F200
PRTS ")\n"
IN  R201
PRTI R201
PRTS "("
INI R200
PRTI R200
PRTS ", "
INF F200
PRTF F200
PRTS ")\n"
PRTS "\n***DONE***\n"
\end{verbatim}

\end{document}
